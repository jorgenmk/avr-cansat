\section*{Forord}

Denne rapporten er dokumentasjon p� arbeidet som er utf�rt i forbindelse med hovedprosjekt ved H�gskolen i S�r-Tr�ndelag. Den tar for seg arbeidet og utfordringene med � bygge en Cansat. CanSat er en sammensmelting av to ord, Can fra engelske ``soda can'' og satellitt. En CanSat er en satellittmodell p� st�rrelse med en 0.33 liters mineralvannsboks, inneholdet betst�r av elektronikk som gj�r den i stand til � innhente atmosf�risk data og sende det til jorden. Oppdraget er gitt av \ac{NAROM} for � utvikle en prototype til bruk i � etablere en nasjonal CanSat-konkurranse.
\\
\\
Rapporten er skrevet av Mats M�ller B�ren, J�rgen Kvalvaag, Gaute Nilsson og Joakim Andre T�nnesen, elektronikkstudenter ved H�gskolen i S�r-Tr�ndelag. Hovedprosjektet ble p�begynt med et m�te for � kartlegge prosjektet 06. januar 2009  og fortsatte til 25. mai 2009. Arbeidet og grunnlaget for rapporten har forg�tt ved H�gskolen i S�r-Tr�ndelag og Norbit lokalisert p� Lade Teknopark i Trondheim.
\\
\\
I hovedsak henvender rapporten seg til personer med bakgrunn innenfor eletronikk og teleteknikk, med vekt p� personell ved H�gskolen i S�r-Tr�ndelag og \ac{NAROM}, men vi inviterer alle som har interesse av rapporten til � komme med tilbakemeldinger.
\\
\\
Vi vil rette en stor takk til Stein �vstedal og og Valdemar Finnanger for gode innspill som veiledere for Cansatgruppen, Atmel for komponenter, Nordic Semiconductor for utviklingsverkt�y og alle andre som har bidrat til prosjektet. Radiokortet er designet etter id� fra Steffen Kirknes. S� en stor takk til Steffen Kirknes og resten av Norbit for hjelp til montering, feils�king og produksjon av radiokretskortet.
