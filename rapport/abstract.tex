\begin{abstract}
Fire elektronikkstudenter startet h�sten 2009 med � konstruere en satellittmodell for den norske romfartsorganisasjonen \ac{NAROM}. Oppgaven b�d p� nye utfordringer med tanke p� det ekstreme milj�et sluttproduktet skulle operere i.

Modellen er laget for � skytes mellom fem og �tte kilometer opp i atmosf�ren. Herfra daler den ned til bakken i fallskjerm. Det er i den fasen elektronikken aktiveres og utf�rer oppgavene sine. Til slutt lander den i havet utenfor And�ya i Nordland, en b�t sendes deretter ut for � hente den til land.

Satelittmodellen er delt opp i fem kort som hver er tildelt sin spesifikke oppgave. Kortene er forbundet via en 20 pins buss. P� denne foreg�r all kommunikasjon i tillegg til at str�mforyning fra batteriet kommer herfra. �verst sitter GPS-kortet for � f� god plassering av mottakerantennen.

Under dette kommer sensorkortet da sensorene m� ha god kontakt med atmosf�ren ved m�ling av lufttrykk, temperatur og  relativ fuktighet. Deretter kommer hovedkortet, her sitter hjernen som styrer aktiviteten til de andre modulene. Hovedkortet har mulighet til � skru av og p� de andre kortene, og har ansvar for at datakommunikasjonen g�r feilfritt.

De resterende kortene er batterikort og radiokort. Radiokortet er plassert nederst slik at senderantennen f�r god sikt mot bakken. Radiokortet sender GPS-posisjon og sensordata til en bakkeenhet. Bakkeenheten kobles til en vanlig datamaskin som lagrer den mottatte informasjonen til disk.

Gruppen ble stilt overfor veldig mange nye utfordringer under arbeidet med oppgaven. Til forskjell fra utstyr produsert for bruk p� landjorden settes det spesielle krav til kvaliteten p� sluttproduktet. N�r raketten f�rst har lettet fra rampen m� alt fungere 100~\%.

Det resulterende produktet ble en ferdigstilt modul�r prototype som samler inn data fra omgivelsene og sender disse til en mottakerenhet i et format som er enkelt � jobbe med for produksjon av statistikk og grafer.

En av de mest spennende aspektene ved prosjektet er mulighetene for videreutvikling og oppgradering. Det modul�re designet gj�r det enkelt � legge til nye enheter, eksempelvis videokamera, styrbar fallskjerm, gyroskop og radar, mulighetene er enorme.

Utstyr produsert til romfartsform�l koster vanligvis sv�rt mye og involverer avansert spesialutstyr. Med dette prosjektet har vi vist at romteknologi ikke trenger � koste en formue. Med litt god fantasi og kloke hoder kommer man langt med hyllevarer og en loddebolt.
\end{abstract}